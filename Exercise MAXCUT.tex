\documentclass[a4paper]{scrartcl}

\usepackage[utf8]{inputenc} %Zeichensatzkodierung
\usepackage[english]{babel} % neue deutsche Rechtschreibung
\usepackage[T1]{fontenc} % vortragstaugliche Schrift
\usepackage{lmodern} % vortragstaugliche Schrift

\usepackage[numbers]{natbib}
\usepackage{fullpage}
\usepackage{amsmath}
\usepackage{amssymb}
\usepackage{amstext}
\usepackage{amsfonts}
\usepackage{amsthm}
\usepackage{mathrsfs}
\usepackage{stmaryrd}
\usepackage{color}
\usepackage[ruled,vlined,linesnumbered]{algorithm2e}
\usepackage{ulsy}
\usepackage{dsfont}

\newcommand{\cupdot}{\mathbin{\mathaccent\cdot\cup}}

\theoremstyle{plain}
\newtheorem{thm}{Theorem}


\title{MAX-CUT Exercise}
\subtitle{Seminar Modern Methods in Combinatorial Optimization}
\date{\today}

\begin{document}
\maketitle
\noindent
Let $G=(V,E)$ be a simple graph with nonnegative edge weights $w_{ij}$. Further assume that $\forall i \in V: \ \{i,i\} \notin E.$\\
An instance of the MAX CUT problem is to find a maximal cut in $G$, i.e. find $S, \bar S \subset V$, s.t. $V = S \cupdot \bar S$ and that
$\sum_{\substack{i\in S \\ j \in\bar S}} w_{ij}$
 is maximized.
 \bigskip

\textbf{Exercise 1:}
Determine the max cut in the following graph:
\vspace{200pt}

\textbf{Exercise 2:}
Formulate MAX CUT as an Integer Quadratic Program.

\bigskip
\noindent
Ask Fin or Stefan if you need hints.

\end{document}