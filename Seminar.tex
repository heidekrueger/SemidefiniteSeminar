\documentclass[11pt]{beamer}
\usetheme{Rochester}

\usepackage[utf8]{inputenc}
\usepackage[german]{babel}
\usepackage{amsmath}
\usepackage{amsfonts}
\usepackage{amssymb}
\usepackage{graphicx}
\usepackage{enumitem}
\setitemize{label=\usebeamerfont*{itemize item}
	\usebeamercolor[fg]{itemize item}
	\usebeamertemplate{itemize item}}

\author{Fin Bauer and Stefan Heidekrüger}
\title{Semidefinite Programming}
%\subtitle{}
%\logo{}
%\institute{}
%\date{}
%\subject{}
%\setbeamercovered{transparent}
\setbeamertemplate{navigation symbols}{}

\begin{document}
	\maketitle
\begin{frame}
	\frametitle{Outline}
	\tableofcontents
\end{frame}
\section{Introduction}	
	\begin{frame}
		\frametitle{What Is Semidefinite Programming?}
	\begin{block}{\vspace*{-3ex}}
		\begin{equation}
		\begin{aligned}
		\min_X \quad\langle C,X \rangle &:= Tr(CX)=\sum_{i=1}^{n}\sum_{j=1}^{n}C_{ij}X_{ij}\\
		\text{s.t.}\quad\langle A_i,X\rangle&\:= b_i\\
		X&\:\succeq 0
		\end{aligned}
		\end{equation}
	\end{block}
	\end{frame}
	\begin{frame}
		\frametitle{An Easy Example}
		\begin{equation*}
		C=\begin{pmatrix}
		1 & 2 \\
		2 & 9 \\
		\end{pmatrix},\; A_1=\begin{pmatrix}
		1 & 0 \\
		0 & 3 \\
		\end{pmatrix},\; A_2=\begin{pmatrix}
		0 & 2 \\
		2 & 3 \\
		\end{pmatrix},\; b_1=11,\; b_2=19
		\end{equation*}
		\pause
		\begin{block}{\vspace*{-3ex}}
		\begin{equation*}
		\begin{aligned}
		\min\quad && x_{11}+2x_{21}+2x_{12}+9x_{22}\\
		\text{s.t.}\quad&& x_{11}+x_{22}= & \:11\\
		&&  2x_{21}+2x_{12}+3x_{22}= & \:19\\
		&& X=\begin{pmatrix}
		x_{11}& x_{12}\\
		x_{21}& x_{22}\\
		\end{pmatrix}\succeq&\: 0
		\end{aligned}
		\end{equation*}
		\end{block}
	\end{frame}
	\begin{frame}
		\frametitle{Have I Ever Seen Semidefinite Programming Before?}
		\begin{itemize}[leftmargin=-0.3cm]
		\item Linear Programming\\
		\begin{columns}
			\begin{column}{0.5\textwidth}
				\begin{block}{Linear Program}
					\scriptsize
					\abovedisplayskip=0pt
					\abovedisplayshortskip=0pt 
					\belowdisplayskip=0pt
					\belowdisplayshortskip=0pt 
					\begin{equation*}
					\begin{aligned}
					\min_x\quad  b_0^Tx+c_0\quad\\
					\text{s.t.}\quad  b_i^Tx+c_i\leq&  \:0\text{,\quad i in 1,...,n}\\
					 x\geq& \:0
					\end{aligned}
					\end{equation*}
				\end{block}
			\end{column}
			\begin{column}{0.5\textwidth}
				\begin{block}{Hint}
					Diagonal Matrix
				\end{block}
			\end{column}
		\end{columns}
		\item Convex Quadratically Constrained Quadratic Programming\\
		\begin{columns}
		\begin{column}{0.5\textwidth}
		\begin{block}{CQCQP}
		\scriptsize
		\abovedisplayskip=0pt
		\abovedisplayshortskip=0pt 
		\belowdisplayskip=0pt
		\belowdisplayshortskip=0pt 
		\begin{equation*}
		\begin{aligned}
		\min\quad &x^TA_0x+b_0^Tx+c_0\\
		\text{s.t.}\quad &x^TA_ix+b_i^Tx+c_i\leq 0\text{,\quad i in 1,...,n}
		\end{aligned}
		\end{equation*}
		\end{block}
		\end{column}
		\begin{column}{0.5\textwidth}
			\begin{block}{Hint}
			\scriptsize
			\begin{equation*}
			\begin{aligned}
			\text{Given }A_i=M_i^TM_i\\
			\text{then }x^TA_ix+b_i^Tx+c_i\leq 0 \\
			\begin{pmatrix}
			I & M_ix\\
			x^TM_i^T & -c_i-b_i^Tx\\
			\end{pmatrix}
			\succeq 0
			\end{aligned}
			\end{equation*}
			\end{block}
		\end{column}
		\end{columns}
		\end{itemize}
	\end{frame}
\section{Some Theory}
\subsection{Duality}
	\begin{frame}
		\frametitle{What's the Dual?}
		\begin{block}{Primal Problem in Standard Form}
			\begin{equation}
			\inf_X\{Tr(CX);Tr(A_iX)=b_i\;(i=1,...,m),\;X\in \mathcal{S}^+_n\}
			\end{equation}
		\end{block}
		\begin{block}{Dual Problem in Standard Form}
			\begin{equation}
			\sup_{y,S}\{b^Ty;\sum_{i=1}^{m}y_iA_i+S=C,\;S\in \mathcal{S}^+_n,y\in\mathbb{R}^m\}
			\end{equation}
		\end{block}
	\end{frame}
	\begin{frame}
		\frametitle{Weak and Strong Duality}
	\end{frame}
	\begin{frame}
		\frametitle{Example with Duality Gap}
	\end{frame}
\subsection{Optimality}
	\begin{frame}
		\frametitle{When is the Solution Optimal?}
		\begin{block}{Optimality Conditions}
			\begin{equation*}
			\begin{aligned}
			Tr(A_iX)&=b_i,\quad X\succeq 0,\quad i=1,...,m\\
			\sum_{i=1}^{m}y_iA_i+S&=C,\quad S\succeq 0\\
			XS&=0
			\end{aligned}
			\end{equation*}
		\end{block}
	\end{frame}
\section{Algorithms}
\subsection{Interior Points}
\begin{frame}
	\frametitle{bla}
\end{frame}
\section{Applications}
\subsection{Lovasz}
\begin{frame}
	\frametitle{bla}
\end{frame}
\subsection{Max Cut}
\begin{frame}
	\frametitle{bla}
\end{frame}
\subsection{Machine Learning}
\begin{frame}
	\frametitle{bla}
\end{frame}
\end{document}